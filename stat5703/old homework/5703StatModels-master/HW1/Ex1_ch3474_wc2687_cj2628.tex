% Options for packages loaded elsewhere
\PassOptionsToPackage{unicode}{hyperref}
\PassOptionsToPackage{hyphens}{url}
%
\documentclass[
]{article}
\usepackage{lmodern}
\usepackage{amssymb,amsmath}
\usepackage{ifxetex,ifluatex}
\ifnum 0\ifxetex 1\fi\ifluatex 1\fi=0 % if pdftex
  \usepackage[T1]{fontenc}
  \usepackage[utf8]{inputenc}
  \usepackage{textcomp} % provide euro and other symbols
\else % if luatex or xetex
  \usepackage{unicode-math}
  \defaultfontfeatures{Scale=MatchLowercase}
  \defaultfontfeatures[\rmfamily]{Ligatures=TeX,Scale=1}
\fi
% Use upquote if available, for straight quotes in verbatim environments
\IfFileExists{upquote.sty}{\usepackage{upquote}}{}
\IfFileExists{microtype.sty}{% use microtype if available
  \usepackage[]{microtype}
  \UseMicrotypeSet[protrusion]{basicmath} % disable protrusion for tt fonts
}{}
\makeatletter
\@ifundefined{KOMAClassName}{% if non-KOMA class
  \IfFileExists{parskip.sty}{%
    \usepackage{parskip}
  }{% else
    \setlength{\parindent}{0pt}
    \setlength{\parskip}{6pt plus 2pt minus 1pt}}
}{% if KOMA class
  \KOMAoptions{parskip=half}}
\makeatother
\usepackage{xcolor}
\IfFileExists{xurl.sty}{\usepackage{xurl}}{} % add URL line breaks if available
\IfFileExists{bookmark.sty}{\usepackage{bookmark}}{\usepackage{hyperref}}
\hypersetup{
  pdftitle={STAT5703 HW1 Ex1},
  pdfauthor={Chao Huang (ch3474), Wancheng Chen (wc2687), Chengchao Jin (cj2628)},
  hidelinks,
  pdfcreator={LaTeX via pandoc}}
\urlstyle{same} % disable monospaced font for URLs
\usepackage[margin=1in]{geometry}
\usepackage{graphicx}
\makeatletter
\def\maxwidth{\ifdim\Gin@nat@width>\linewidth\linewidth\else\Gin@nat@width\fi}
\def\maxheight{\ifdim\Gin@nat@height>\textheight\textheight\else\Gin@nat@height\fi}
\makeatother
% Scale images if necessary, so that they will not overflow the page
% margins by default, and it is still possible to overwrite the defaults
% using explicit options in \includegraphics[width, height, ...]{}
\setkeys{Gin}{width=\maxwidth,height=\maxheight,keepaspectratio}
% Set default figure placement to htbp
\makeatletter
\def\fps@figure{htbp}
\makeatother
\setlength{\emergencystretch}{3em} % prevent overfull lines
\providecommand{\tightlist}{%
  \setlength{\itemsep}{0pt}\setlength{\parskip}{0pt}}
\setcounter{secnumdepth}{-\maxdimen} % remove section numbering
% https://github.com/rstudio/rmarkdown/issues/337
\let\rmarkdownfootnote\footnote%
\def\footnote{\protect\rmarkdownfootnote}

% https://github.com/rstudio/rmarkdown/pull/252
\usepackage{titling}
\setlength{\droptitle}{-2em}

\pretitle{\vspace{\droptitle}\centering\huge}
\posttitle{\par}

\preauthor{\centering\large\emph}
\postauthor{\par}

\predate{\centering\large\emph}
\postdate{\par}

\title{STAT5703 HW1 Ex1}
\author{Chao Huang (ch3474), Wancheng Chen (wc2687), Chengchao Jin
(cj2628)}
\date{}

\begin{document}
\maketitle

\hypertarget{exercise-1.}{%
\subsection{Exercise 1.}\label{exercise-1.}}

\hypertarget{question-1.}{%
\paragraph{Question 1.}\label{question-1.}}

To calculate \(p^{th}\), we need to find \(Q_D(p)\) such that
\(P(D\leq Q_D(p))=p\). Then, the previous function can be transformed to

\[\int_{0}^{Q_D(p)}\lambda e^{-\lambda D}dD=1-e^{-\lambda Q_D(p)}=p\]

So, from this equation, \(Q_D(p)\) can be expressed by

\[Q_D(p)=-\frac{1}{\lambda}\ln{(1-p)}\]

\hypertarget{question-2.}{%
\paragraph{Question 2.}\label{question-2.}}

From question(a), we already obtain the equation for \(Q_D(p)\) which is
\(Q_D(p)=-\frac{1}{\lambda}\ln{(1-p)}\). Then, to find the MLE of
\(Q_D(p)\), we can find the MLE of \(\lambda\) first and then replace
\(\lambda\) with its Maximum Likelihood Estimator \(\hat\lambda^{MLE}\).
\(D_1,...,D_n\) are i.i.d. Exponential random variables with parameter
\(\lambda\), the log-likelihood function is

\[\ell (\lambda ;D_1,...,D_n)=n\ln{\lambda}-\sum_{i=1}^n\lambda D_i\]
The MLE \(\hat\lambda^{MLE}\) is

\[\hat\lambda^{MLE}=\frac{1}{\bar{D_n}}\]

and

\[Q_D(p)^{MLE}=-\frac{1}{\hat\lambda^{MLE}}\ln{(1-p)}=-\bar{D_n}\ln{(1-p)}\]

\hypertarget{question-3.}{%
\paragraph{Question 3.}\label{question-3.}}

\(D_1,...,D_n\overset{i.i.d.}{\sim}Exp(\lambda)\). Then the CLT tells us
that
\[\sqrt{n}(\bar{D_n}-\mu)\xrightarrow[n\rightarrow \infty]{\mathcal{D}}\mathcal{N}(0,\sigma^2)\]
Hence, by Delta Method we can get,
\[\sqrt{n}(Q_D(p)+\frac{\ln{(1-p)}}{\lambda})\xrightarrow[n\rightarrow \infty]{\mathcal{D}}\mathcal{N}(0,\frac{(\ln{(1-p)})^2}{\lambda^2})\]
Then, for \(approximate\ (1-\alpha)\)-confidence interval,

\[L(D)=-\bar{D_n}\ln{(1-p)}-\frac{z_{1-\alpha/2}\times\ln{(1-p)}}{\lambda\sqrt{n}}\]

\[R(D)=-\bar{D_n}\ln{(1-p)}+\frac{z_{1-\alpha/2}\times\ln{(1-p)}}{\lambda\sqrt{n}}\]
So, the \(approximate\ (1-\alpha)\)-confidence interval for \(Q_D(p)\)
is
\([-\bar{D_n}\ln{(1-p)}-\frac{z_{1-\alpha/2}\times\ln{(1-p)}}{\lambda\sqrt{n}},-\bar{D_n}\ln{(1-p)}+\frac{z_{1-\alpha/2}\times\ln{(1-p)}}{\lambda\sqrt{n}}]\)

\hypertarget{question-4.}{%
\paragraph{Question 4.}\label{question-4.}}

We know that if \(D_1,...,D_n\) are independent exponential random
variables with parameter \(\lambda\), then
\[\lambda\bar{D_n}\sim \Gamma(n,n)\] So, \(\lambda\bar{D_n}\) is
independent of the parameter \(\lambda\), which means it is an exact
pivot. To construct an exact confidence interval of the median, we can
first transform \(\lambda\bar{D_n}\) to \(\chi^2\) distribution. Then,
\[2n\lambda\bar{D_n}\sim \chi^2_{2n}\] Hence, for any
\(\alpha\in(0,1)\),
\[P(\chi^2_{1-\alpha/2,2n}<2n\lambda\bar{D_n}<\chi^2_{\alpha/2,2n})=1-\alpha\]
Since \(Q_D(0.5)=-\bar{D_n}\ln{0.5}\), then

\[P(-\frac{\ln{0.5}\chi^2_{\alpha/2,2n}}{2n\lambda}<Q_D(0.5)<-\frac{\ln{0.5}\chi^2_{1-\alpha/2,2n}}{2n\lambda})=1-\alpha\]

Hence, the \((1-\alpha)\) exact confidence interval of the median is,
\[Q_D(0.5)\in[-\frac{\ln{0.5}\chi^2_{\alpha/2,2n}}{2n\lambda},-\frac{\ln{0.5}\chi^2_{1-\alpha/2,2n}}{2n\lambda}]\]

\end{document}
